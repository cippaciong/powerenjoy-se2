\newcommand{\code}[1]{\texttt{#1}}
\newcommand{\issue}[3][?]{
    \paragraph{[#1] #2} \mbox{}\\ #3
}

\section{List of issues}

Here is a comprehensive list of all issues we have found according to the
checklist reported in the assignment.
For the sake of clarity we decided to write down only relevant and 
meaningful points and to omit the other ones.

\subsection{Naming Conventions}

All naming convenctions are respected.

\subsection{Indentation}

\issue[9]{Tabs}
	Four spaces are consistently used to indent except at line 208, where there is a tab.

\subsection{Braces}

\issue[10]{Bracing style}
	The ``Kernighan and Ritchie'' style is consistently used.

\issue[11]{If with one statement}
	At line 61 there is an if with only one statement that isn't surrounded by curly braces:
	\begin{lstlisting}[basicstyle=\small\ttfamily,columns=fullflexible]
	if (userLogin == null) return;
	\end{lstlisting}
	Instead it's better to use this style:
	\begin{lstlisting}[basicstyle=\small\ttfamily,columns=fullflexible]
	if (userLogin == null) {
		return;
	}
	\end{lstlisting}

\subsection{Initialization and Declarations}

\issue[28]{Visibility}
    \code{VisitHandler} class has only static methods and variables but doesn't
    have a private constructor.

\issue[33]{Declarations appear at the beginning of blocks}
\begin{itemize}
    \item
        \code{visit} variable declared at line 78 is not at the beginning of the block.\\
        \begin{lstlisting}[basicstyle=\small\ttfamily,columns=fullflexible]
    [...]
74  } catch (GenericEntityException e) {
75    Debug.logError(e, "Could not update visitor: ", module);
76  }
77
78  GenericValue visit = getVisit(session);
    [...]
        \end{lstlisting}
    \item
        \code{visitor} variable declared at line 163 is not at the beginning of the block.\\
        \begin{lstlisting}[basicstyle=\small\ttfamily,columns=fullflexible]
160  visit.set("clientUser", session.getAttribute("_CLIENT_REMOTE_USER_"));
161  
162  // get the visitorId
163  GenericValue visitor = (GenericValue) session.getAttribute("visitor");
164  if (visitor != null) {
        \end{lstlisting}
\end{itemize}

\subsection{Computation, Comparisons and Assignments}

\issue[44]{"Brutish programming"}
    The following three lines contain a triple way if statement inside the method
    invocations itself. This makes the whole line pretty long and the code less readable.
    It would probably have been better to extract this operation in separate methods
    with a better interface.

    \begin{lstlisting}[basicstyle=\small\ttfamily,columns=fullflexible]
150  if (initialRequest != null) visit.set("initialRequest",
        initialRequest.length() > 250 ? initialRequest.substring(0, 250)
        : initialRequest);
151  if (initialReferrer != null) visit.set("initialReferrer",
        initialReferrer.length() > 250 ? initialReferrer.substring(0, 250)
        : initialReferrer);
152  if (initialUserAgent != null) visit.set("initialUserAgent",
        initialUserAgent.length() > 250 ? initialUserAgent.substring(0, 250)
        : initialUserAgent);
    \end{lstlisting}

\subsection{Exceptions}

\issue[53]{Check that the appropriate action are taken for each catch block.}
    While we think that exceptions are caught properly, some log messages saved
within this blocks are a bit vague. 

    \begin{lstlisting}[basicstyle=\small\ttfamily,columns=fullflexible]
74  Debug.logError(e, "Could not update visitor: ", module);
94  Debug.logError(e, "Could not update visit: ", module);
190 Debug.logError(e, "Could not create new visit:", module);
247 Debug.logError(e, "Could not create new visitor:", module);
    \end{lstlisting}

    Furthermore, a message with a context similar to the previous ones is logged as warning instead of error.
    \begin{lstlisting}[basicstyle=\small\ttfamily,columns=fullflexible]
280 Debug.logWarning(new Exception(), "Could not find or create the visitor...", module);
    \end{lstlisting}
% subsection authenticate (end)
