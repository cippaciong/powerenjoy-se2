\section{Introduction}
This document report on the quality status of assigned code extracts using the checklist for Java Code Inspection reported on the Appendix of \textit{Code Inspection Assignment Task Description} document.\\
Code inspection is an examination of computer source code. In particular it aims to find mistakes overlooked during the development phase in order to improve the quality of the software and the development' skills. Reviews are done in various forms such as pair programming, informal walkthroughs, and formal inspections.\\

This document is divided into n main parts:
\begin{description}
	\item[Section 1: Introduction]
		this first section gives a short introduction to the Code Inspection document
	\item[Section 2: Assigned classes]
		this section is to state the classes that we are going to analyze
	\item[Section 3: Functional role] 
		this section contains the description of the functional role we have identified for the classes specifies in the previous section
	\item[Section 4: List of issues] 
		in this section we report the fragment of code that do not fulfill some points in the checklist
	\item[Section 7: Working Hours] 
		this sections contains the result of our effort, quantified in the number of hours we have needed in order to develop this document. 
	\item[Section 8: References] 
		a list of all the references we took into account when writing this document.
\end{description}