\pagebreak
\section{Integration strategy}

\subsection{Entry criteria}
% Specify the criteria that must be met before integration testing of specific elements may begin (e.g. functions must have been unit teste)
There are some conditions on the project development that must be met before starting the integration testing. 

\begin{itemize}
	\item \textit{Requirements Analysis and Specification Document} and the \textit{Design Document} must have been written, in order to have a complete picture of the interactions between the different components of the system and their functionalities.
	\item All the functions and individual modules have already been tested through a unit testing process and all unit test bugs have been fixed.
	% dobbiamo aggiungere anche le percentuali di completamento che ogni componente deve avere prima di iniziare l'integration testing??
\end{itemize}


\subsection{Elements to be Integrated}
% Identify the components to be integrated, refer to your design document to identify such components in a way that is consistent with your design


\subsection{Integration Testing Strategy}
% Describe the integration testing approach (top-down, bottom-up, functional groupings, etc.) and the rationale for the choosing that approach.
The strategy that will be adopted is the so-called \textbf{bottom-up} approach.
This approach comes natural with the structure of our components. They have been
designed with the clear scope in mind to make them atomic and as decoupled as possible.
We can take advantage of these design choices now and start our integration tests
on the single component. Once we are confident that the component behaves
well within its subcomponents and with the external services, we can proceed testing
it's integration with another component and so on.
As a short example we can consider one of the simplest use cases, the creation of a new user.
The full sequence of actions that should be performed in order tu fulfill the request
would be something like this:
\lstset{basicstyle=\ttfamily\footnotesize,breaklines=true}
\begin{lstlisting}
HTTP Request
    |
    v
    API Gateway
        |
        v
        Dispatch Message
                |
                v 
                User Handler
                     |
                     v
                     Validate
                         |
                         v
                         Insert in the DB.
\end{lstlisting}
The whole operation can be seen as a sort of pipeline where each subcomponents
receive the input from the previous one and passes its output to the next.
Following the bottom-up approach we can start testing the integration between the
database and the code whose responsibility is to interact with it. Once we are sure
that everything is working as expected we move up one step and test the previous integration.
In the end we should be able to test the whole "pipeline" and make sure that the
goal of creating a user is achieved.

These simple test operations will be iterated over every single component and then
over groups of components interacting with each other.

\subsection{Sequence of Component/Function Integration}
Thanks to the fact that our architecture is loosely coupled, we have a certain flexibility
choosing the order in which tests should be performed.
We start testing the integration of subcomponents in a depth-first manner and
move upward in the stack until we find a dependency with another component.
Alternatively following what we could call a breadth-first approach,
we can decide to complete the integration tests of every component and
move up in the stack only when all of them are done.
We will prefer the first choice because we think it provides a higher degree of
flexibility. During the development process we can start testing and prototype
some functionalities even if we have not yet developed the other components.
