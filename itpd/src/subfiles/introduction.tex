\section{Introduction}

\subsection{Revision history}
This is version 1.0 of our Integration Test Plan Document.
In this paragraph we will list changes of each future version of the document. \\

\begin{tabular}{|l|l|l|l|}
	\hline
	\textbf{Version}	& \textbf{Date}	& \textbf{Authors}	& \textbf{Summary}\\
	\hline
	1.0					& 15/01/217		& Patrizia Porati, Tommaso Sardelli	& First version\\
	\hline
	2.0					& 7/02/217		& Patrizia Porati, Tommaso Sardelli	& Minor fixes\\
\hline
\end{tabular}

\subsection{Purpose and Scope}
This is the Integration Test Plan Document (ITPD). This document aims at describing the plan to accomplish the integration testing of the components that compose the PowerEnJoy application.
Integration testing is the intermediate task between the unit testing, that tests sections of code and individual modules, and the system testing, that tests the complete system.
The purpose of this level of testing is to expose faults in the interaction between integrated units.

\subsection{List of Definitions and Abbreviations}
The following are used throughout the document:
\begin{description}
	\item [RASD] Requirement Analysis and Specification Document
	\item [DD] Design Document
	\item [ITPD] Integration Test Plan Document
	\item [Unit testing] a software testing method used to test individual units of source code, single modules, functions and procedures
	% ADD definitions
\end{description}

\subsection{List of Reference Documents}
\begin{itemize}
	\item The project descritption document: \textit{Assignments AA 2016-2017.pdf}
	\item PowerEnJoy Requirement Analysis and Specification Document: \textit{rasd.pdf}
	\item PowerEnJoy Design Document: \textit{dd.pdf}
	% ADD documentation of any tool you plan to use for testing
\end{itemize}

