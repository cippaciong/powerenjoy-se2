\pagebreak
\subsection{Actors identification}

\newcommand{\actor}[2]{
	\item \textbf{#1:} #2}

\begin{itemize}
	\actor {Guest}{a guest, as we already mentioned in the glossary section, is someone who hasn't sign up. He/she can only visit the homepage and sign up.}
	
	\actor{Registered user}{a registered user, or simply a user, is a person that has already registered in the system. He/she has a profile with his/her personal information and, after logging in, he/she has the capability of using all the services that the application provides.}
	
	\actor{Logged user}{a logged user is a registered user that has already logged in. When a user is logged, he/she can:
		\begin{itemize}
			\item visualize his/her profile page, personal information and previous rides and reservations;
			\item update personal information;
			\item search and reserve available cars;
			\item unlock the reserved car.
		\end{itemize}
	}

	\actor{Administrator}{a ``PowerEnJoy'' employee whose task is to manage cars, safe areas, users and administrators. An administrator can:
		\begin{itemize}
			\item see the list of all users;
			\item add and delete users;
			\item see the details of a selected user, update his/her information, change his/her status and modify payment details;
			\item get the list of all cars;
			\item add and delete cars;
			\item see the details of a selected car and change its status;
			\item see the list of all safe areas and power grid stations;
			\item add and delete a safe area or a power grid station;
			\item add a new administrator.
		\end{itemize}}
	
\end{itemize}