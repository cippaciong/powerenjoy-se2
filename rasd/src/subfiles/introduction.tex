\section{Introduction}

\subsection{Purpose}
    This is the Requirement Analysis and Specification Document (RASD from now on).
    The aim of this document is to show the functional and non-functional requirements
    of the system-to-be, based on several important aspects:
    the needs expressed by the stakeholders, the constraints which it is subject to,
    the typical scenarios that will happen after its deployment.
    The targeted audience is mainly made of software engineers and developers who have
    to actually develop the service here described.
    We want to make clear from the beginning that in this document we are not going to
    discuss what will be implemented or how. We are just going to collect and analyze
    all the customer requirements and to provide a general idea of how the product should
    look in the end and how the users should intecact with it.

\subsection{Scope}
    The task we are asked to complete is the definition of a software system to manage a car sharing
    service composed by electric cars. This is going to be a brand new system without any
    legacy software or data to deal with. The idea is that users can register to our platform
    using an internet connected device (computer, smartphone, etc.) and then they are able to
    look for available cars in a certain area. 
    One of the available cars can be reserved and picked up in not more then one hour; at that
    point the user is billed for the time he's using the car.
    Finally, the system can apply discounts or penalties.

\subsection{Stakeholders}
    The main stakeholder is the \textbf{PowerEnJoy} company, whose aim is to provide a service
    that is profitable for them but at the same time is useful for all the people living in the
    city and helps reduce the pollution thanks to the electric engines.

\subsection{Definitions}
    \begin{itemize}
        \item \textbf{Agents}
        \begin{itemize}
            \item \textbf{Guest}: A person who is not registered to the platform. He can either
                register or browse the public website
            \item \textbf{RegisteredUser}: A registered person that has full access to the platform.
            \item \textbf{User}: See "RegisteredUser".
            \item \textbf{Passenger}: A person that is taken by a RegisteredUser as his
                his passenger during a ride. It doesn't matter if such person is registered or not.
        \end{itemize}
        \item \textbf{Car}: An electric car owned by PowerEnJoy.		
        \item \textbf{Reservation}: A one hour lasting booking of a car performed by a 
            single user.
        \item \textbf{Ride}: A ride is what follows a reservation when the user picks up the
            car in time. It begins with the car unlocking, it ends when the car is parked and
            locked and it keeps track of the user who drove the car and the time of car usage.
        \item \textbf{SafeArea}: A PowerEnJoy parking slot where the User can leave the car
            at the end of the ride.
        \item \textbf{PowerGridStation}: A special SafeArea with an electric outlet to charge
            the car battery.
    \end{itemize}
