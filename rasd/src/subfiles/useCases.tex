\pagebreak
\section{Use Cases}
%\label{sec:req}

% Definition of UseCase environment
\newenvironment{UseCase}[5]{
	\paragraph{Partecipating actors:} #1
	\paragraph{Entry condition:} #2
	\paragraph{Flow of events:}
		\begin{itemize} 
			#3 
		\end{itemize}
	\paragraph{Exit condition:} #4
	\paragraph{Exceptions:}
		\begin{itemize}
			#5
		\end{itemize}
	}{}

\newcommand{\Event}[1]{
	\item #1
}

\newcommand{\Exc}[2]{
	\item \textbf{#1:} #2

}


\subsection{Registration}
\begin{UseCase}
{Guest: a guest is whoever visits the website}
{This use case starts when the guest access the homepage of the web application and clicks on ``\textit{Sign up}''.}
{
	\Event{The guest clicks on ``\textit{Sign up}''}
	\Event{The guest fills out the form entering all required information:
		\begin{itemize}
			\item name
			\item surname
			\item address
			\item e-mail
			\item mobile phone number
			\item driving license number
			\item credit card number
	\end{itemize}}
	\Event{The guest checks the checkbox ``\textit{I have read and agree t PowerEnJoy Terms of Use and Privacy Policy}''}
	\Event{The guest clicks on ``\textit{Confirm}''}
	\Event{The system verifies that the user's credit card is valid}
	\Event{The system verifies that the user's driving license is valid}
	\Event{The system stores the new data in the users database}
	\Event{The system displays a confirmation message, informing the new user that the registration has been successfully completed}
	\Event{The system shows the new user the log in page}
	\Event{The system sends the user an e-mail containing his/her credentials (username and password)}
}
{This use case terminates when the registration is successfully completed and the new user receives the mail with his/her credentials.}
{
	\Exc{The guest is already a registered user}{If this exception occurs, the system displays the error message: ``\textit{ERROR: you are already registered!}'' and the application goes back to the homepage.}
	\Exc{The user's credit car isn't valid}{If this exception occurs, the system displays the error message: ``\textit{ERROR: your credit card is not valid!}'' and the application goes back to the homepage.}
	\Exc{The user's driving license is not valid}{If this exception occurs, the system displays the error message: ``\textit{ERROR: your driving license is not valid!}'' and the application goes back to the homepage.}
	\Exc{The user doesn't fill all the fields in the registration form}{If this exception occurs, the system displays the error message: ``\textit{ERROR: all the fileds has to be filled!}'' and the application goes back to the page where the registration form is shown.}
}
\end{UseCase}


\subsection{Log in}
\begin{UseCase}
	{Registered user: a registered user is a guest that has already sign up.}
	{This use case starts when the registered user, that has already received the e-mail with his/her credentials, clicks on ``\textit{Log in}'' from the homepage of the website.}
	{
		\Event{The registered user clicks on ``\textit{Log in}''}
		\Event{The registered user enters his/her username}
		\Event{The registered user enters his/her password}
		\Event{The registered user clicks on ``\textit{Confirm}''}
		\Event{The system checks the inserted username}
		\Event{The system checks the inserted password}
		\Event{The system shows the user's personal profile page}
	}
	{This use case terminates when the log in is successfully completed and the new user access his/her personal area.}
	{
		\Exc{The inserted username is incorrect}{If this exception occurs, the system displays the error message: ``\textit{ERROR: the inserted username is wrong!}'' and the application goes back to the homepage.}
		\Exc{The inserted password is incorrect}{If this exception occurs, the system displays the error message: ``\textit{ERROR: the inserted password is wrong!}'' and the application goes back to the log in page}
	}
\end{UseCase}

\subsection{Password recovery}
\begin{UseCase}
	{Registered user: a registered user is a guest that has already sign up.}
	{This use case starts when the user forgot the password and clicks on ``\textit{Send a new password}'' from the log in page}
	{
		\Event{The registered user is on the log in page and clicks on ``\textit{Send a new password}''}
		\Event{The system sends the user an e-mail with a new password}
		\Event{The system displays the log in page}
		\Event{The user receive the e-mail with the new password}
	}
	{This use case terminates when the user receive the new password by e-mail.}
	{No exceptions.}
\end{UseCase}

\subsection{Reserve a car}
\begin{UseCase}
	{Registered user: a registered user is a guest that has already sign up.}
	{This use case starts when, after logging in, the registered user, clicks on ``\textit{Reserve a car}'' from his/her personal profile page.}
	{
		\Event{The registered user clicks on ``\textit{Reserve a car}''}
		\Event{The system shows the reservation page}
		\Event{The user selects where to search the car: choosing if use his/her current location or a specified address (in this case he/she also has to write an address)}
		\Event{The user selects a maximum distance for the car research}
		\Event{The user clicks on ``\textit{Search cars}''}
		\Event{The system searches available cars within the maximum distance indicated from the given location}
		\Event{The system shows the list of available cars}
		\Event{The user selects one of the cars in the list}
		\Event{The user clicks on ``\textit{Reserve}''}
		\Event{The system displays a confirmation message containing the details of the reservation (time of the reservation, license plate, position, distance from the given location \ldots)}
		\Event{The system displays the user's personal profile page}
	}
	{This use case terminates when the confirmation message is shown and the user is redirected on his/her profile page.}
	{
		\Exc{The user doesn't select a location for the car research}{If this exception occurs, the system displays the error message: ``\textit{ERROR: you have to select a location for the car research!}'' and the application goes back to the reservation page.}
		\Exc{The user writes an inexistent address}{If this exception occurs, the system displays the error message: ``\textit{ERROR: the inserted address doesn't exist!}'' and the application goes back to the reservation page.}
		\Exc{The user writes an invalid value for the maximum distance}{If this exception occurs, the system displays the error message: ``\textit{ERROR: the inserted distance isn't valid!}'' and the application goes back to the reservation page.}
		\Exc{There aren't cars in the selected area}{If this exception occurs, the system displays the error message: ``\textit{ERROR: there are no cars in the selected area! Please change the maximum distance or the selected location.}'' and the application goes back to the reservation page.}
	}
\end{UseCase}