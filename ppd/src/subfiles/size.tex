\pagebreak
\section{Project size, cost and effort estimation}

In this chapter we provide the the expected size, the cost and effort estimation of the ``PowerEnJoy'' application.

\subsection{Size estimation: Function Points}

We apply the \textbf{Function Points} approach, based on the main functionalities of the application, to evaluate the size of the project and estimate the line of code to be written. Then  we can proceed with the estimation of the effort and the duration applying the rules defined by the COCOMO II approach. \\
\\
We use this table to evaluate the weight for every function:

\begin{center}
	\begin{tabular}{|l|c|c|c|}
		\hline
		\textbf{Function types} 	& \textbf{Simple} 	& \textbf{Medium} 	& \textbf{Complex} \\
		\hline \hline
		N. Internal Files 	& 7 	& 10 	& 15 \\
		\hline
		N. External Files 	& 5 	& 7 	& 10 \\
		\hline
		N. Inputs 	& 3 	& 4 	& 6 \\
		\hline
		N. Output 	& 4 	& 5 	& 7 \\
		\hline
		N. Inquiries 	& 3 	& 4 	& 6 \\	
		\hline
	\end{tabular}
\end{center}

\subsubsection{Internal Logic Files (ILFs) (\textit{entities of the system})}
``PowerEnJoy'' application relies on a number of ILFs to store the information it needs to offer the required functionalities. \\
The system has to store information about \textit{Registered Users} data: id, name, surname, email, password, address, user status (active, insolvent, inactive), phone number, credit card number and driving license number. \\
It also stores \textit{Administrators} information: name, email and password. \\
\textit{Reservations} are stored in a dedicated table to store an id, the id of the user who reserved the car, the identifier of the reserved car and the date and the time of the reservation. \\
An other table containing information that the applications needs is the table of the \textit{Cars}, where for each car are stored an id, the license plate number, the level of the battery, the position, the status (available, unavailable, reserved, in use), locked (boolean: true if the car is locked, false otherwise), charging status (boolean: true if the car is plugged to the power grid, false otherwise). \\
Then there is the \textit{Ride} table, where data concerning the ride is stored: id, start position, end position, duration, number of passengers, total price and the identifier of the related reservation. \\ 
There is also a table for \textit{Discounts} types containing an identifier, the name, the percentage and a description. \\ 
Each ride can be related to zero or more discounts and each discount can be related to zero or more ride, so we need a table where store this information (\textit{Discounts Applied}): this table only contains the ride identifier and the ride identifier. \\
\textit{Payments} information is also stored in a dedicated table containing the identifier number, the price, the time and the date, the registered user's identifier, the identifier of the related ride. \\
The system needs to know where and which are the \textit{safe areas}, so this information is stored in a dedicated table, that contains: id number, address, longitude, latitude, total places, free places and power grid (boolean: true if it's a PowerGridStation, false if it's a simple SafeArea).

\begin{center}
	\begin{tabular}{|l|l|c|}
		\hline
		\textbf{ILFs} 	& \textbf{Complexity} 	& \textbf{FPs} \\
		\hline
		Registered Users 	& high 	& 15 \\
		Administrators 	& simple 	& 7 \\
		Reservations 	& medium 	& 10 \\
		Cars 	& medium	& 10 \\
		Rides 	& high 	& 15 \\
		Discounts 	& simple 	& 7 \\
		Discounts Applied 	& medium 	& 10 \\
		Payments 	& high	& 15 \\
		Safe Areas  	& medium	& 10 \\
		\hline \hline
		\textbf{Total} 	& 	& 99 \\
		\hline
	\end{tabular}
\end{center}

\subsubsection{External Logic Files (ELFs) (\textit{communication between different software})}

	\todo{Tommy \textless 3}
	
\begin{center}
	\begin{tabular}{|l|l|c|}
		\hline
		\textbf{ELFs} 	& \textbf{Complexity} 	& \textbf{FPs} \\
		\hline
		Boh 	& boh 	& boh \\

		\hline \hline
		\textbf{Total} 	& 	& boh \\
		\hline
	\end{tabular}
\end{center}


\subsubsection{External Inputs (EIs)}

``PowerEnJoy'' web application supports many kind of interactions with different actors.
Here we summarize the EIs grouping them by actor's category:
\begin{itemize}
	\item The application interacts with \underline{Guests}:
		\begin{itemize}
			\item \textbf{registration}: this operation has a \textit{medium} complexity because it requires some checks on the validity of the inputs (it contributes 4 FPs).
		\end{itemize}
	\item The application interacts with \underline{RegisteredUsers}:
		\begin{itemize}
			\item \textbf{login}: it is a \textit{simple} operation that requires only email and password (it contributes 3 FPs);
			\item \textbf{logout}: this is a \textit{simple} operation that involve only the UserHandler component, so we adopt the simple weight for it (it contributes 3 FPs);
			\item \textbf{password recovery}: this operation requires some steps in order to create a new password and update the database, so we adopt the \textit{medium} weight for it (it contributes 4 FPs);
			\item \textbf{update personal information}: this is a \textit{medium} complexity operation because it requires to update some fields and check their validity (it contributes 4 FPs);
			\item \textbf{reserve a car}: this operation requires a some difficult steps (like `search available cars', `check that the user hasn't already an active reservation') to reach the goal, so it's an operation with \textit{high} complexity (it contributes 6 FPs);
			\item \textbf{unlock the reserved car}: this operation requires some steps (like `check user distance from the car' and `change car status'), so it is an \textit{high} complexity operation (it contributes 6 FPs).
		\end{itemize}
	\item The application interacts also with \underline{Administrators}:
		\begin{itemize}
			\item \textbf{insert, delete and update PowerEnJoy areas (SafeAreas and PowerGridStations)}: these operations require to check the validity of the inputs and to update the database, so we adopt the \textit{medium} weight for them (it contributes 4 FPs each);
			\item \textbf{create new administrators}: this operation has a \textit{medium} complexity because it requires to check the inputs and to update the database (it contributes 4 FPs);
			\item \textbf{insert, delete and update electric cars}: these operations require to check the validity of the input, check and sometimes change the status of the car, provide some car information (like the actual position) and update the database, so we adopt the \textit{high} weight for them (it contributes 6 FPs each);
			\item \textbf{insert, delete and update users}: this operation has a \textit{medium} complexity because it requires to check the inputs and to update the database (it contributes 4 FPs each);
		\end{itemize}
\end{itemize}

\begin{center}
	\begin{tabular}{|l|l|c|}
		\hline
		\textbf{EIs} 	& \textbf{Complexity} 	& \textbf{FPs} \\
		\hline
		Registration 	& medium 	& 4 \\
		Login 	& simple 	& 3 \\
		Logout 	& simple 	& 3 \\
		Password recovery 	& medium 	& 4 \\
		Update personal information 	& medium 	& 4 \\
		Reserve a car 	& high 	& 6 \\
		Unlock the reserved car 	& high 	& 6 \\
		Insert, delete and update areas 	& medium 	& 3 x 4 \\
		Add a new admin 	& medium 	& 4 \\
		Insert, delete and update cars 	& high 	& 3 x 6 \\
		Insert, delete and update users 	& medium 	& 3 x 4 \\		
		\hline \hline
		\textbf{Total} 	& 	& 76 \\
		\hline
	\end{tabular}
\end{center}

\subsubsection{External Inquiries (EQs) (\textit{actions that required data retrieval from the database})}

	\todo{}

\subsubsection{External Outputs (EOs) (\textit{things that are created from the system for the user})}

	\todo{}

\subsubsection{Overall estimation}

	\todo{}
