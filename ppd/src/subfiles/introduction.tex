\section{Introduction}

\subsection{Revision History}

	This is version 2.0 of our Project Plan Document.
	In this paragraph we will list changes of each future version of the document. \\
	
	\begin{tabular}{|l|l|l|l|}
		\hline
		\textbf{Version}	& \textbf{Date}	& \textbf{Authors}	& \textbf{Summary}\\
		\hline
		1.0					& 22/01/2017		& Patrizia Porati, Tommaso Sardelli	& First version\\
		\hline
		2.0					& 7/02/2017		& Patrizia Porati, Tommaso Sardelli	& Minor fixes\\
		\hline
	\end{tabular}

\subsection{Purpose and Scope}

	This is the Project Plan Document for the ``PowerEnJoy'' web application. The aim of this document is to analyze the compleXity of the application and identify the main tasks needed to accomplish the all project in order to provide the cost and the effort estimation. In this way we define the required budjet, the resources allocation, the schedule of the activities and identify possible risks. \\
	In this first chapter we are going to provide a brief introduction to our document. \\
	In the second chapter we apply two of the main algorithmic estimation techniques (Function points and COCOMO  II ) used to estimate size in term of line of code, the cost and the effort required to develop the entire application. \\
	In chapter three we schedule all the activities starting from the requirements identification and the fulfillment of the plan documents to the implementation and testing. \\
	Then we allocate some tasks to each group member. \\
	At the and we identify possible risks that can occur and draw some general conclusions.
	
	
\subsection{List of Definitions and Abbreviations}

	The following are used throughout the document:
	\begin{description}
		\item [RASD] Requirement Analysis and Specification Document
		\item [DD] Design Document
		\item [ITPD] Integration Test Plan Document
		\item [PPD] Project Plan Document
		\item [PP] Project Plan
		\item [SLOC] Source Line Of Code
		\item [LOC] Line Of Code
		\item [External Input (EI)] elementary operation used to elaborate data coming from the external environment
		\item [External Output (EO)] elementary operation that generates data for the external environment
		\item [External Inquiry (EQ)] elementary operation that involves inputs and outputs of the system
		\item [External Logic File (ELF)] homogeneous set of data used by the application but generated and maintained by other applications
		\item [Function Points (FP)] a technique used to evaluate the effort needed to design and develop custom software applications
		\item [Internal Logic File (ILF)] homogeneous set of data used and managed by the application itself
		\item [Project Planning] phase of the Software Project Management that aims at identifying tasks, estimating and scheduling the project development and assigning people to tasks
		\item [Risk] potential problem that can occur
		\item [Task] an activity that must be completed in order to achieve the project goal
		% ADD definitions
	\end{description}


\subsection{Reference Documents}
\begin{itemize}
	\item The project descritption document: Assignments AA 2016-2017.pdf
	\item PowerEnJoy Requirement Analysis and Specification Document: rasd.pdf
	\item PowerEnJoy Design Document: dd.pdf
	\item PowerEnJoy Integration Test Document: itpd.pdf
	\item COCOMO II - Model Definition Manual:\\ http://csse.usc.edu/csse/research/COCOMOII/cocomo2000.0/CII\_modelman2000.0.pdf
	\item Function Point Languages Table: http://www.qsm.com/resources/function-point-languages-table
\end{itemize}
	