\newcounter{SequenceCounter}
\newcommand{\sequence}[2]{
	\refstepcounter{SequenceCounter}
	\begin{figure}[H]
		\includegraphics[width=\textwidth,height=\textheight,keepaspectratio]{#1.eps}
		\caption*{Sequence Diagram \arabic{SequenceCounter}: #2}
		\label{sequence:#1}
	\end{figure}
}


\pagebreak
\subsection{Runtime view}
In this paragraph we are going to show some sequence diagrams related to the use cases identified in the RASD document, in order to explain how the components of the system interacts with each other and with the actors.

\subsubsection{Registration}
\sequence{registration}{The registration}

\pagebreak
\subsubsection{Log in}
\sequence{login}{The log in}

\pagebreak
\subsubsection{Recover the password}
\sequence{password}{The recovery of the password}

\pagebreak
\subsubsection{Reserve a car}
\sequence{reservation}{Reservation of a car}

\pagebreak
\subsubsection{Unlock the reserved car}
\sequence{unlock}{Unlock the reserved car}

\pagebreak
\subsubsection{Update personal information}
In order to enhance understanding, in thise sequence diagram we decided to omit the router component because its only task is to switch the request to the userHandler.
\sequence{update}{Update personal information}

\pagebreak
\subsubsection{Show the details of the ride}
\sequence{details}{Show the details of the ride}

\pagebreak
\subsubsection{End of the ride}
\sequence{end}{End of the ride}

\pagebreak
\subsubsection{Add a new car}
In order to enhance understanding, in thise sequence diagram we decided to omit the router component because its only task is to switch the request to the carHandler.
\sequence{addCar}{Add a new car to the system}