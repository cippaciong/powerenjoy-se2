\section{Introduction}

\subsection{Purpose}
This document describes the architecture underlying PowerEnJoy,
starting from the specifications and requirements described in the RASD document.
We are going to describe our choices about software and hardware.
Software components, will be presented showing relations and interactions between them,
as well as architectural styles and patterns chosen. Hardware architecture will be presented
showing how the software will be deployed on it.
We will use UML diagrams in increasing detail as a standard language to explain concepts.
 
\subsection{Scope}
This document provides additional and more detailed information extending what has been presented in the RASD,
but from a different point of view. For this reason, the Design Document and the RASD 
document developed earlier have to be both considered in order to have a general view of our project 
in all its aspects. 
The intended audience of this specification include project managers, but above all software developers 
that are going to implement our application in the future. 
Within the document we are going to examine architectural choices and technologies employed
with different levels of details, but we opted to omit implementations details that looked 
too specific to us. Examples of those details are programming languages or
software to use across the system.
We believe that the eventual implementer should (and should be able to) implement the devised
architecture using the tools and languages he prefers or knows best.
 
\subsection{Definitions, acronyms and abbreviations}
The following are used throughout the document:
\begin{description}
\item[RASD] Requirements Analysis and Specification Document.
\item[DD] Design Document.
\item[REST] REpresentational State Transfer.
\item[IaaS] Infrastructure as a Service.
\item[RDBMS] Relational Database Management System.
\item[Reverse Proxy] Proxy server that retrieves resources on behalf of a client
    from one or more servers.
\item[Load Balancer] Element handling the distribution of workloads across
    multiple computing resources
\end{description}
