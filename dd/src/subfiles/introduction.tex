\section{Introduction}

\subsection{Purpose}
This document describes the architecture underlying PowerEnJoy,
starting from the specifications and requirements described in the RASD document.
We are going to describe our choices about software and hardware.
Software components, will be presented showing relations and interactions between them,
as well as architectural styles and patterns chosen. Hardware architecture will be presented
showing how the software will be deployed on it.
We will use UML diagrams in increasing detail as a standard language to explain concepts.
 
\subsection{Scope}
The intended audience of this specification include project managers, but above all software developers 
that are going to implement our application in the future. 
Within the document we are going to examine architectural choices and technologies employed
with different levels of details, but we opted to omit implementations details that looked 
too specific to us. Examples of those details are programming languages or
software to use across the system.
We believe that the eventual implementer should (and should be able to) implement the devised
architecture using the tools and languages he prefers or knows best.
 
\subsection{Definitions, acronyms and abbreviations}
The following are used throughout the document:
\begin{description}
\item[RASD] Requirements Analysis and Specification Document.
\item[DD] Design Document.
\item[REST] REpresentational State Transfer.
\item[IaaS] Infrastructure as a Service.
\item[RDBMS] Relational Database Management System.
\item[Reverse Proxy] Proxy server that retrieves resources on behalf of a client
    from one or more servers.
\item[Load Balancer] Element handling the distribution of workloads across
    multiple computing resources
\end{description}

 
\subsection{Reference documents }
This document provides additional and more detailed information extending what has been
presented in the RAS Document. They both concur to provide a general overview and understanding
of the purposes and the architectural details of our platform.
As for the organization of the chapters and the paragraphs, we followed the structure and the 
subdivision into paragraphs of the template for the design document made available by our professor. 
Finally, we also helped ourselves consulting the IEEE Standards for the Design and the Architecture 
Descriptions. 
 
 
\subsection{Document structure}
Our Design Document is divided into eight main parts: 
\begin{description}
    \item[Section 1: Introduction]
        this first section gives an introduction to the Design Document, specifying the purpose and the 
        scope of the document, the glossary (containing all the terms, definitions, acronyms or 
        abbreviations used) and the documents we referred to in order to develop this paper. 
    \item[Section 2: Architectural Design]
        this section is the core of the document. It gives information about the design and the 
        architecture of our application, defining all the software and hardware components 
        characterizing the system and how they can interact between each other.
    \item[Section 3: Algorithm Design]
        this section contains the description of the main algorithms used in our system. 
    \item[Section 4: User Interface Design]
        this section contains a detailed representation of the users interfaces. Through specific and 
        detailed mockups. 
    \item[Section 5: Requirements Traceability] 
        this section contains the correspondences between the requirements we have listed in the 
        RASD document and the components we have identified during the design and architectural 
        analysis. 
    \item[Section 6: Used Tools] 
        in this section, all the tools we have used in order to develop this document are listed. 
    \item[Section 7: Working Hours] 
        this sections contains the result of our effort, quantified in the number of hours we have 
        needed in order to develop this document. 
    \item[Section 8: References] 
        a list of all the references we took into account when writing this document.
\end{description}
